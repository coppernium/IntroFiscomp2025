\chapter*{Motivação}

A análise de funções por meio de métodos numéricos é uma das ferramentas centrais da Física Computacional. 
Derivadas, integrais e equações algébricas estão no coração da formulação matemática de quase todos os fenômenos físicos, 
mas nem sempre as soluções exatas são acessíveis. Nesse cenário, técnicas aproximadas permitem transformar problemas complexos 
em algoritmos implementáveis em computador, abrindo caminho para simulações que seriam impossíveis de resolver de forma puramente analítica.

A derivação numérica, por exemplo, fornece meios de obter informações sobre taxas de variação mesmo quando só conhecemos valores discretos 
de uma função. Já a quadratura numérica possibilita calcular áreas, probabilidades e fluxos em situações onde não há primitiva conhecida. 
O estudo de raízes, por sua vez, conecta-se diretamente à busca por estados estacionários e soluções de equações transcendentais que emergem 
em mecânica, termodinâmica e teoria de campos.

Além da aplicação direta, a implementação desses métodos exige compreender limitações de precisão, erros de truncamento e estabilidade numérica. 
Esse exercício não apenas fortalece a intuição sobre os algoritmos, mas também ilustra a importância da expansão em série de Taylor como base 
conceitual unificadora.

Assim, este trabalho busca integrar teoria e prática computacional, explorando como procedimentos aparentemente simples, diferenças finitas, 
regras de quadratura e métodos iterativos de raízes, constituem a base para simulações mais sofisticadas. A motivação central está em compreender 
que dominar esses métodos é um passo fundamental para investigar problemas reais da física, onde a complexidade frequentemente supera as ferramentas 
analíticas tradicionais.
