\chapter*{Tarefa - A}

\section*{Código}


\vspace*{1\baselineskip}

\begin{figure}[H]
\centering
\caption{Função principal do código.}
\centering

\begin{lstlisting}
        program main
		implicit real*8 (a-h,o-z)
		call euler_crommer()
		call euler()
		end program main
\end{lstlisting}

\caption*{Fonte: Compilado pelo Autor.}
\label{fig:tarefa 1 - função principal do código}
\end{figure}

\vspace*{1\baselineskip}

\begin{figure}[H]
\centering
\caption{Implementação do método de Euler-Cromer.}
\centering

\begin{lstlisting}
subroutine euler_crommer()
implicit real*8 (a-h,o-z)
parameter (imax = 1e3)
dimension w(0:imax), th(0:imax)

g = 9.81d0
m = 1d0
rl = 9.81d0
dt = 0.01d0
pi = acos(-1d0)

C       Condições Iniciais
w(0) = 0d0
th(0) = pi/6



do i = 0, imax-1
w(i+1) = w(i) - (g/rl)*sin(th(i))*dt
th(i+1) = th(i) + w(i+1)*dt

end do

open(unit=1,file='saida-1-12694394.txt')
do i = 0, imax

! Trem do angulo
if (th(i+1) .gt. 2*pi) then
th(i+1) = th(i+1) - 2*pi
else if (th(i+1) .lt. 0d0) then
th(i+1) = th(i+1) + 2*pi
end if

write(1,*) i*dt, w(i), th(i)
end do
close(1)


open(unit=2,file='saida-2-12694394.txt')
do i = 0, imax
E_kin = 0.5d0*m*((rl*w(i))**2) 
E_pot = -m*g*rl*cos(th(i))


write(2,*) i*dt, E_kin,E_pot,E_kin+E_pot
end do
close(2)

end subroutine euler_crommer

\end{lstlisting}

\caption*{Fonte: Compilado pelo Autor.}
\label{fig:tarefa 1 - Euler Cromer código}
\end{figure}



\vspace*{1\baselineskip}

\begin{figure}[H]
\centering
\caption{Implementação do método de Euler.}
\centering

\begin{lstlisting}
subroutine euler()
implicit real*8 (a-h,o-z)
parameter (imax = 1e4)
dimension w(0:imax), th(0:imax)

g = 9.81d0
m = 1d0
rl = 9.81d0
dt = 0.01d0
pi = acos(-1d0)

C       Condições Iniciais
w(0) = 0d0
th(0) = pi/6d0



do i = 0, imax-1
w(i+1) = w(i) - (g/rl) * sin(th(i)) * dt
th(i+1) = th(i) + w(i) * dt
end do

open(unit=3,file='saida-3-12694394.txt')
do i = 0, imax

! Trem dos angulos
if (th(i+1) .gt. 2*pi) then
th(i+1) = th(i+1) - 2*pi
else if (th(i+1) .lt. 0d0) then
th(i+1) = th(i+1) + 2*pi
end if

write(3,*) i*dt, w(i), th(i)
end do
close(3)


open(unit=4,file='saida-4-12694394.txt')
do i = 0, imax
E_kin = 0.5d0*m*((rl*w(i))**2) 
E_pot = -m*g*rl*cos(th(i))

write(4,*) i*dt, E_kin,E_pot,E_kin+E_pot
end do
close(4)

end subroutine euler

\end{lstlisting}

\caption*{Fonte: Compilado pelo Autor.}
\label{fig:tarefa 1 - Derivadas}
\end{figure}



\newpage
\section*{Descrição do código}

O programa \texttt{main} tem como objetivo comparar dois métodos numéricos 
de integração aplicados ao movimento de um pêndulo simples: o método de 
\textit{Euler} e o método de \textit{Euler-Cromer}. Ambos são implementados 
em subrotinas independentes e chamados a partir do programa principal.  

\vspace*{1\baselineskip}
\noindent
No início do código, é utilizado o comando:

\begin{lstlisting}
	implicit real*8 (a-h,o-z)
\end{lstlisting}

\noindent
que define todas as variáveis cujos nomes começam com as letras de 
\textbf{a} a \textbf{h} e \textbf{o} a \textbf{z} como números reais de 
dupla precisão.  
Em seguida, o programa principal executa as duas subrotinas:

\begin{lstlisting}
	call euler_crommer()
	call euler()
\end{lstlisting}

\noindent
As duas subrotinas calculam numericamente a evolução temporal da velocidade 
angular ($\omega$) e do ângulo ($\theta$) de um pêndulo simples, de 
comprimento $r_l$ e massa $m$, sob ação da gravidade $g$.  

\subsection*{Subrotina \texttt{euler\_crommer}}

Na subrotina \texttt{euler\_crommer}, o número máximo de iterações é 
definido por \texttt{imax = 1e3}, e são declarados os vetores 
\texttt{w(0:imax)} e \texttt{th(0:imax)}, que armazenam respectivamente a 
velocidade angular e o ângulo.  

Os parâmetros físicos e o passo temporal são definidos como:

\begin{lstlisting}
	g = 9.81d0
	m = 1d0
	rl = 9.81d0
	dt = 0.01d0
	pi = acos(-1d0)
\end{lstlisting}

\noindent
As condições iniciais são estabelecidas como $\omega(0) = 0$ e 
$\theta(0) = \pi/6$.  
O método de Euler-Cromer é então aplicado no laço:

\begin{lstlisting}
	do i = 0, imax-1
	w(i+1) = w(i) - (g/rl)*th(i)*dt
	th(i+1) = th(i) + w(i+1)*dt
	end do
\end{lstlisting}

\noindent
onde o valor atualizado da velocidade angular é utilizado imediatamente 
no cálculo do ângulo, característica principal do método de Euler-Cromer, 
que tende a conservar melhor a energia mecânica em sistemas oscilatórios.  

Os resultados da evolução temporal de $\omega$ e $\theta$ são gravados no 
arquivo \texttt{saida-1-12694394.txt}, enquanto as energias cinética, 
potencial e total são salvas em \texttt{saida-2-12694394.txt}.  
Antes da escrita, o código realiza uma correção para manter o ângulo 
$\theta$ dentro do intervalo $[0, 2\pi]$.  

As expressões utilizadas para as energias são:

\begin{align*}
	E_{\text{cin}} &= \frac{1}{2} m (r_l \, \omega)^2, \\
	E_{\text{pot}} &= -m g r_l \cos(\theta).
\end{align*}

\noindent
A soma dessas quantidades fornece a energia total 
$E_{\text{tot}} = E_{\text{cin}} + E_{\text{pot}}$.

\subsection*{Subrotina \texttt{euler}}

A subrotina \texttt{euler} implementa o método de Euler simples, para fins 
de comparação com o método anterior.  
São utilizados os mesmos parâmetros físicos, mas com um número maior de 
iterações (\texttt{imax = 1e4}), a fim de aumentar a resolução temporal.  

O esquema numérico empregado é dado por:

\begin{lstlisting}
	do i = 0, imax-1
	w(i+1) = w(i) - (g/rl)*th(i)*dt
	th(i+1) = th(i) + w(i)*dt
	end do
\end{lstlisting}

\noindent
Diferentemente do método de Euler-Cromer, aqui o novo valor de $\theta$ é 
calculado usando a velocidade \textit{anterior} $w(i)$, o que leva a uma 
maior variação da energia total ao longo do tempo.  

Os resultados para $\omega$ e $\theta$ são gravados no arquivo 
\texttt{saida-3-12694394.txt}, e as energias correspondentes em 
\texttt{saida-4-12694394.txt}.  

Assim como na subrotina anterior, é aplicado o ajuste de periodicidade para 
o ângulo e são calculadas as energias cinética, potencial e total com as 
mesmas expressões.  

\subsection*{Resumo}

Em síntese, o código compara numericamente os métodos de Euler e 
Euler-Cromer na simulação de um pêndulo simples, permitindo observar 
diferenças na conservação da energia ao longo do tempo.  
Cada método gera dois arquivos de saída: um contendo os parâmetros 
dinâmicos ($\omega$ e $\theta$) e outro contendo as energias calculadas em 
cada instante.


\section*{Resultados}


\begin{figure}[H]
	\centering
	\caption{Video simulação do pendulo.}
	\centering
	\includemovie[autoplay]{12cm}{7cm}{videos/tarefa-A/saida-1-12694394.mp4}
	\caption*{Fonte: Compilado pelo Autor.}
	\label{fig:tarefa 1 - Video 1}
\end{figure}

\begin{figure}[H]
\centering
\caption{Energia método Euler-Cromer}
\centering
\includegraphics[width=16cm]{images/tarefa-A/fig1.png}
\caption*{Fonte: Compilado pelo Autor.}
\label{fig:tarefa A - Resultado energia Euler-Cromer}
\end{figure}

\begin{figure}[H]
\centering
\caption{Energia método Euler}
\centering
\includegraphics[width=16cm]{images/tarefa-A/fig2.png}
\caption*{Fonte: Compilado pelo Autor.}
\label{fig:tarefa A - Resultado energia Euler}
\end{figure}