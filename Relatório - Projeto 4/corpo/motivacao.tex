\chapter*{Motivação}


O estudo do movimento oscilatório é um tema fundamental na Física, 
pois está presente em diversos fenômenos naturais e tecnológicos. 
Desde o pêndulo simples até sistemas mais complexos, compreender 
como um corpo oscila e como essas oscilações se comportam com o 
tempo é essencial para entender vários tipos de movimento.

Neste projeto, o objetivo é analisar o comportamento do pêndulo 
simples e suas variações usando métodos numéricos, como o de Euler 
e o de Euler-Cromer. A ideia é observar como diferentes 
aproximações e parâmetros influenciam o movimento, além de comparar 
os resultados obtidos numericamente com as soluções analíticas 
conhecidas.

Além disso, o projeto permite explorar casos mais realistas, 
incluindo efeitos dissipativos e forças externas, que tornam o 
sistema mais interessante e imprevisível. Nesses casos, o pêndulo 
pode apresentar desde movimentos periódicos até comportamentos 
caóticos, dependendo das condições iniciais e dos parâmetros 
escolhidos. Estudar esse tipo de sistema ajuda a entender melhor 
como surgem fenômenos caóticos na natureza e mostra a importância 
das simulações computacionais para investigar situações em que o 
cálculo analítico não é suficiente.