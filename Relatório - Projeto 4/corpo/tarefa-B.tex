\chapter*{Tarefa - B}


\section*{Código}


\vspace*{1\baselineskip}

\begin{figure}[H]
\centering
\caption{Função principal do código.}
\centering

\begin{lstlisting}
        program main
        call euler_crommer()
        end program main
\end{lstlisting}

\caption*{Fonte: Compilado pelo Autor.}
\label{fig:tarefa B - função principal do código}
\end{figure}

\vspace*{1\baselineskip}

\begin{figure}[H]
\centering
\caption{Implementação do método de Euler-Cromer.}
\centering

\begin{lstlisting}

        subroutine euler_crommer()
        parameter (imax=1e4)
        implicit real*8(a-h,o-z)
        dimension th(0:imax),w(0:imax)

C       Constantes
        rl = 9.81d0
        g = 9.81d0
        m = 1d0
        y = 0.050d0 !Gamma
        F0 = 0.50d0 !Força externa
        ome = 0.75d0 ! Frequência da força externa
        pi = acos(-1d0)
        dt = 0.04d0

C       Valores iniciais
        w(0) = 0
        th(0) = pi/3d0

C       Cálculo

        do i = 0,imax-1
                F_ex = -y*w(i) + F0*sin(ome*i*dt)
                w(i+1) = w(i)-(g/rl)*sin(th(i))*dt + F_ex*dt
                th(i+1) = th(i) + w(i+1)*dt
        end do

C       Salva a posição
        open(unit=1,file='saida-1-12694394.txt')
        do i =0,imax
        ! Trem das posições
        if (th(i+1) .GT. 2*pi) then
                th(i+1) = th(i+1) - 2*pi
        else if (th(i+1) .LT. 0 ) then
                th(i+1) = th(i+1) + 2*pi
        end if

                write(1,7) i*dt, w(i), th(i)
        end do
7       format(F12.6,F12.6,F12.6)
        close(1)
        end subroutine euler_crommer

\end{lstlisting}

\caption*{Fonte: Compilado pelo Autor.}
\label{fig:tarefa B - Euler Cromer código}
\end{figure}



\newpage\section*{Descrição do código}

O programa \texttt{main} tem como objetivo simular numericamente o movimento 
de um pêndulo simples sujeito a atrito viscoso e a uma força externa 
periódica, utilizando o método de integração de \textit{Euler-Cromer}.  
O cálculo é implementado na subrotina \texttt{euler\_crommer}, chamada a 
partir do programa principal.  

\vspace*{1\baselineskip}
\noindent
A diretiva

\begin{lstlisting}
implicit real*8 (a-h,o-z)
\end{lstlisting}

\noindent
define todas as variáveis cujos nomes iniciam com as letras de 
\textbf{a} a \textbf{h} e \textbf{o} a \textbf{z} como números reais de 
dupla precisão, garantindo maior precisão nos cálculos numéricos.  

\subsection*{Subrotina \texttt{euler\_crommer}}

A subrotina \texttt{euler\_crommer} executa o cálculo principal da 
simulação.  
Primeiramente, são definidos o número máximo de iterações 
(\texttt{imax = 1e4}) e os vetores \texttt{w(0:imax)} e 
\texttt{th(0:imax)}, que armazenam respectivamente a velocidade angular 
($\omega$) e o ângulo ($\theta$) do pêndulo em cada passo de tempo.  

As constantes físicas e parâmetros do sistema são definidos como:

\begin{lstlisting}
rl = 9.81d0      ! Comprimento do pêndulo
g  = 9.81d0      ! Aceleração da gravidade
m  = 1d0         ! Massa
y  = 0.050d0     ! Coeficiente de amortecimento (Gamma)
F0 = 0.50d0      ! Amplitude da força externa
ome = 0.75d0     ! Frequência angular da força externa
dt = 0.04d0      ! Passo temporal
pi = acos(-1d0)  ! Valor de pi
\end{lstlisting}

\noindent
As condições iniciais são fixadas como $\omega(0) = 0$ e 
$\theta(0) = \pi/3$.  

O método de Euler-Cromer é então aplicado dentro de um laço de 
iterações que atualiza a velocidade angular e o ângulo a cada passo de 
tempo.  
A força externa e o termo de amortecimento são incluídos na equação de 
movimento, resultando no seguinte esquema numérico:

\begin{lstlisting}
do i = 0, imax-1
    F_ex = -y*w(i) + F0*sin(ome*i*dt)
    w(i+1) = w(i) - (g/rl)*sin(th(i))*dt + F_ex*dt
    th(i+1) = th(i) + w(i+1)*dt
end do
\end{lstlisting}

\noindent
O termo \texttt{F\_ex} representa a soma da força de amortecimento 
($-y\omega$) e da força externa periódica 
($F_0 \sin(\omega_{\text{ext}} t)$).  
O uso do método de Euler-Cromer garante maior estabilidade e melhor 
conservação de energia do sistema em comparação com o método de Euler 
tradicional, especialmente em sistemas oscilatórios.  

\subsection*{Saída dos resultados}

Após o cálculo, os resultados da simulação são gravados no arquivo 
\texttt{saida-1-12694394.txt}.  
Cada linha do arquivo contém o tempo ($t = i \cdot \Delta t$), a velocidade 
angular $\omega$ e o ângulo $\theta$, conforme o formato:

\begin{lstlisting}
7 format(F12.6,F12.6,F12.6)
\end{lstlisting}

\noindent
Antes da escrita, o programa corrige o ângulo $\theta$ para mantê-lo dentro 
do intervalo $[0, 2\pi]$, utilizando a verificação:

\begin{lstlisting}
if (th(i+1) .GT. 2*pi) then
    th(i+1) = th(i+1) - 2*pi
else if (th(i+1) .LT. 0) then
    th(i+1) = th(i+1) + 2*pi
end if
\end{lstlisting}

\noindent
Por fim, o arquivo é fechado com o comando \texttt{close(1)}.  

\subsection*{Resumo}

Em resumo, o código realiza a integração numérica das equações de movimento 
de um pêndulo forçado e amortecido pelo método de Euler-Cromer.  
O programa permite estudar o comportamento dinâmico do sistema sob 
diferentes condições de força e amortecimento, sendo útil para análises de 
regimes oscilatórios, ressonância e comportamento caótico em pêndulos 
não-lineares.


\section*{Resultados}


\begin{figure}[H]
	\centering
	\caption{Enunciado da Tarefa 1.}
	\centering
	\includemovie[autoplay]{12cm}{7cm}{videos/tarefa-A/saida-1-12694394.mp4}
	\caption*{Fonte: Compilado pelo Autor.}
	\label{fig:tarefa 1 - Video 1}
\end{figure}
