\chapter*{Tarefa - E}

\section*{Enunciado}
\begin{figure}[H]
\centering
\caption{Enunciado da Tarefa E.}
\centering
\includegraphics[width=16cm]{images/tarefa-E/enunciado-tarefaE.png}
\caption*{Fonte: Compilado pelo Autor.}
\label{fig:tarefa E - Enunciado}
\end{figure}

\section*{Código}


\vspace*{1\baselineskip}

\begin{figure}[H]
\centering
\caption{Função principal do código.}
\centering

\begin{lstlisting}
        program main
            implicit real*8 (a-h,o-z)
            call poincare_secao()
        end program main
\end{lstlisting}

\caption*{Fonte: Compilado pelo Autor.}
\label{fig:tarefa E - função principal do código}
\end{figure}

\vspace*{1\baselineskip}

\begin{figure}[H]
\centering
\caption{Implementação do método de Euler-Cromer.}
\centering

\begin{lstlisting}

        subroutine poincare_secao()
            parameter(imax=1e6)
            implicit real*8 (a-h,o-z)
            dimension w(0:imax), th(0:imax)

            pi = acos(-1d0)
            g  = 9.81d0
            rl = 9.81d0
            dt = 0.04d0
            c1 = 0.05d0
            c2 = 1.2d0
            c3 = 0.666d0

            w(0)  = 0.0d0
            th(0) = pi/60d0

            open(unit=10,file='saida-1-12694394.txt')
            write(10,9)

            do i = 0, imax-1
                t  = i*dt
                rr = c3*t
                F = -c1*w(i) + c2*sin(rr)
                w(i+1)  = w(i) - (g/rl)*sin(th(i))*dt + F*dt
                th(i+1) = th(i) + w(i+1)*dt

                if (th(i+1) .GT. 2*pi) then
                    th(i+1) = th(i+1) - 2*pi
                else if (th(i+1) .LT. 0d0) then
                    th(i+1) = th(i+1) + 2*pi
                end if

                if (abs(mod(c3*t, pi)) .LT. (c3*dt/2d0)) then
                    write(10,8) w(i+1), th(i+1)
                end if
            end do

            close(10)
8           format(F16.8,',',F16.8)
9           format('omega,theta')
            return
        end

\end{lstlisting}

\caption*{Fonte: Compilado pelo Autor.}
\label{fig:tarefa E - Euler Cromer código}
\end{figure}

\section*{Descrição do código}

O programa \texttt{main} tem como objetivo chamar a subrotina 
\texttt{poincare\_secao}, responsável por calcular a seção de Poincaré de 
um pêndulo amortecido e forçado, utilizando o método numérico de 
\textit{Euler-Cromer}.  
A simulação permite analisar o comportamento dinâmico do sistema em 
regimes periódicos e caóticos, registrando apenas os pontos específicos 
que compõem a seção de Poincaré.

\vspace*{1\baselineskip}
\noindent
Toda a integração numérica e o armazenamento dos resultados são realizados 
dentro da subrotina \texttt{poincare\_secao}.  

\subsection*{Parâmetros e variáveis}

Na subrotina, define-se o número máximo de iterações 
\texttt{imax = 1×10\textsuperscript{6}} e são criados dois vetores 
principais: \texttt{w} e \texttt{th}, que armazenam, respectivamente, a 
velocidade angular ($\omega$) e o ângulo ($\theta$) do pêndulo ao longo do 
tempo.

As constantes físicas e parâmetros da força externa são definidos como:

\begin{lstlisting}
pi = acos(-1d0)   ! Valor de pi
g  = 9.81d0       ! Aceleração da gravidade
rl = 9.81d0       ! Comprimento do pêndulo
dt = 0.04d0       ! Passo temporal
c1 = 0.05d0       ! Coeficiente de amortecimento (gamma)
c2 = 1.2d0        ! Amplitude da força externa (F_0)
c3 = 0.666d0      ! Frequência angular da força externa
\end{lstlisting}

\noindent
As condições iniciais do sistema são:
\begin{lstlisting}
w(0)  = 0.0d0
th(0) = pi/60d0
\end{lstlisting}

\noindent
Esses valores correspondem a um pêndulo inicialmente em repouso, com um 
pequeno deslocamento angular inicial, sujeito à ação de uma força externa 
periódica e a um termo de amortecimento viscoso.

\subsection*{Integração pelo método de Euler-Cromer}

A integração numérica é realizada dentro de um laço que percorre de 
\texttt{i = 0} até \texttt{imax - 1}.  
Em cada iteração, calcula-se a força resultante sobre o pêndulo, composta 
pelos termos de amortecimento e força externa:

\begin{lstlisting}
t  = i*dt
rr = c3*t
F  = -c1*w(i) + c2*sin(rr)
\end{lstlisting}

\noindent
As equações diferenciais são integradas pelo método de 
\textit{Euler-Cromer}, que atualiza primeiro a velocidade angular e, em 
seguida, o ângulo:

\begin{lstlisting}
w(i+1)  = w(i) - (g/rl)*sin(th(i))*dt + F*dt
th(i+1) = th(i) + w(i+1)*dt
\end{lstlisting}

\noindent
Esse método é mais estável para sistemas oscilatórios, pois utiliza o 
valor atualizado da velocidade para calcular a posição.

\subsection*{Correção angular e critério da seção de Poincaré}

Após a atualização, o ângulo é ajustado para permanecer dentro do 
intervalo $[0, 2\pi]$, evitando que valores excedam esse limite devido à 
integração acumulada:

\begin{lstlisting}
if (th(i+1) .GT. 2*pi) then
    th(i+1) = th(i+1) - 2*pi
else if (th(i+1) .LT. 0d0) then
    th(i+1) = th(i+1) + 2*pi
end if
\end{lstlisting}

\noindent
A cada passo, o código verifica se o instante atual $t$ corresponde 
aproximadamente a um múltiplo de $\pi / \omega_{força}$, ou seja, se o 
sistema se encontra na mesma fase da força externa.  
Quando essa condição é satisfeita, o par 
$(\omega, \theta)$ é registrado no arquivo de saída, compondo a 
\textit{seção de Poincaré}:

\begin{lstlisting}
if (abs(mod(c3*t, pi)) .LT. (c3*dt/2d0)) then
    write(10,8) w(i+1), th(i+1)
end if
\end{lstlisting}

\subsection*{Armazenamento dos resultados}

Os pontos da seção de Poincaré são salvos no arquivo 
\texttt{saida-1-12694394.txt}, em formato CSV, com as colunas 
\texttt{omega} e \texttt{theta}.  
O formato de escrita utilizado é:

\begin{lstlisting}
8 format(F16.8,',',F16.8)
\end{lstlisting}

\noindent
O arquivo é finalizado com o comando \texttt{close(10)} após o término 
das iterações.

\subsection*{Resumo}

Em síntese, o código implementa a simulação de um pêndulo amortecido e 
forçado, aplicando o método de Euler-Cromer e registrando apenas os pontos 
pertencentes à seção de Poincaré.  
Esses pontos representam o estado do sistema em instantes sincronizados 
com a força externa, permitindo a visualização de órbitas periódicas, 
quase-periódicas e caóticas no espaço de fases.  
Esse tipo de análise é fundamental no estudo de sistemas não lineares e 
fenômenos caóticos em mecânica clássica.

\section*{Resultados}

\begin{figure}[H]
\centering
\caption{Resultado obtidos para $F_0$ = 1.2}
\centering
\includegraphics[width=16cm]{images/tarefa-E/saida-1-12694394.png}
\caption*{Fonte: Compilado pelo Autor.}
\label{fig:tarefa E - Resultado 1}
\end{figure}

\begin{figure}[H]
\centering
\caption{Resultado obtidos para $F_0$ = 0.5}
\centering
\includegraphics[width=16cm]{images/tarefa-E/saida-2-12694394.png}
\caption*{Fonte: Compilado pelo Autor.}
\label{fig:tarefa E - Resultado 2}
\end{figure}
