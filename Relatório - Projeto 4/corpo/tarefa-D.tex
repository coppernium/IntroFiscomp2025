\chapter*{Tarefa - D}

\section*{Enunciado}

\begin{figure}[H]
\centering
\caption{Enunciado da Tarefa D.}
\centering
\includegraphics[width=16cm]{images/tarefa-D/enunciado-tarefaD.png}
\caption*{Fonte: Compilado pelo Autor.}
\label{fig:tarefa D - Enunciado}
\end{figure}


\section*{Código}


\vspace*{1\baselineskip}

\begin{figure}[H]
\centering
\caption{Função principal do código.}
\centering

\begin{lstlisting}
        program main
            implicit real*8 (a-h,o-z)
            call poincare()
        end program main
\end{lstlisting}

\caption*{Fonte: Compilado pelo Autor.}
\label{fig:tarefa D - função principal do código}
\end{figure}

\vspace*{1\baselineskip}

\begin{figure}[H]
\centering
\caption{Implementação do método de Euler-Cromer.}
\centering

\begin{lstlisting}

        subroutine poincare()
            parameter(imax=1e6)
            implicit real*8 (a-h,o-z)
            dimension w(0:imax), th(0:imax)
            dimension ival(0:imax)
C           Constantes
            pi = acos(-1d0)
            g = 9.81d0
            rl = 9.81d0
            dt = 0.04d0
            c1 = 0.05d0 ! Fator de amorteciemento (gamma)
            c2 = 1.2d0 ! F_0
            c3 = 0.666d0 ! Frequencia angular da força extena

C           Valores iniciais
            w(0) = 0
            th(0) = pi/60d0
            ival(0) = 0
C           Realiza a simulação
            open(unit=7,file='saida-2-12694394.txt')
            write(7,9)

            do i = 0,imax-1
                rr = c3*i*dt
                F = -c1*w(i) + c2*sin(rr)
                w(i+1) = w(i) -(g/rl)*sin(th(i))*dt +F*dt
                th(i+1) = th(i) + w(i+1)*dt
                
                if (abs(mod(rr, pi)) .LE. 1d-3) then
                    ival(i+1) = 1
                    write(7,8) w(i+1),th(i+1)
                else
                    ival(i+1) = 0
                end if

            end do

            close(7)
8           format(F16.8,',',F16.8)
9           format('omega,theta')

\end{lstlisting}

\caption*{Fonte: Compilado pelo Autor.}
\label{fig:tarefa D - Euler Cromer código 1}
\end{figure}

\begin{figure}[H]
\centering
\caption{Implementação do método de Euler-Cromer.}
\centering

\begin{lstlisting}

C           Salva os dados
            open(unit=1,file='saida-1-12694394.txt')
                write(1,3)
            do i = 0,imax-1
            ! Aqui eu tenho que eu tenho que fazer o trem do angulo
            if (th(i+1).GT. 2*pi) then
                th(i+1) = th(i+1) -2*pi
            else if (th(i+1) .LT. 0d0) then
                th(i+1) = th(i+1) + 2*pi
            end if


                write(1,2) ival(i),dt*i, w(i),th(i)
            end do
2           format(I2,3(',',F16.8))
3           format('poincare,temp,omega,theta')
            close(1)



        end subroutine poincare
\end{lstlisting}

\caption*{Fonte: Compilado pelo Autor.}
\label{fig:tarefa D - Euler Cromer código}
\end{figure}

\section*{Descrição do código}

O programa \texttt{main} tem como função principal chamar a subrotina 
\texttt{poincare}, responsável por simular o movimento de um pêndulo 
forçado e amortecido por meio do método numérico de 
\textit{Euler-Cromer}.  
O objetivo da simulação é gerar o mapa de Poincaré do sistema, 
permitindo a análise do comportamento dinâmico e a identificação 
de regimes periódicos ou caóticos.

\vspace*{1\baselineskip}
\noindent
Toda a parte computacional é implementada dentro da subrotina 
\texttt{poincare}, que realiza tanto a integração temporal das equações 
de movimento quanto o registro dos dados em arquivos de saída.  

\subsection*{Parâmetros e variáveis}

Na subrotina \texttt{poincare}, é definido o número máximo de iterações 
\texttt{imax = 1×10\textsuperscript{6}}, e são criados três vetores 
principais:  
\texttt{w} (velocidades angulares), \texttt{th} (ângulos) e 
\texttt{ival} (indicador lógico que identifica os pontos de Poincaré).  

As constantes e parâmetros físicos do sistema são definidos como:

\begin{lstlisting}
pi = acos(-1d0)   ! Valor de pi
g  = 9.81d0       ! Aceleração da gravidade
rl = 9.81d0       ! Comprimento do pêndulo
dt = 0.04d0       ! Passo temporal
c1 = 0.05d0       ! Coeficiente de amortecimento (gamma)
c2 = 1.2d0        ! Amplitude da força externa (F0)
c3 = 0.666d0      ! Frequência angular da força externa
\end{lstlisting}

\noindent
As condições iniciais são:
\begin{lstlisting}
w(0)  = 0
th(0) = pi/60d0
ival(0) = 0
\end{lstlisting}

\noindent
Esses valores definem um pêndulo inicialmente quase vertical e sem 
velocidade angular, sujeito a uma força externa periódica e a um 
amortecimento viscoso.

\subsection*{Integração pelo método de Euler-Cromer}

A simulação é realizada dentro de um laço que percorre todas as 
iterações de $i = 0$ até \texttt{imax - 1}.  
Em cada passo, calcula-se a força total atuante sobre o pêndulo, composta 
pelo termo de amortecimento e pela força externa periódica:

\begin{lstlisting}
rr = c3*i*dt
F  = -c1*w(i) + c2*sin(rr)
\end{lstlisting}

\noindent
Em seguida, as equações diferenciais do sistema são integradas utilizando 
o método de Euler-Cromer:

\begin{lstlisting}
w(i+1)  = w(i) - (g/rl)*sin(th(i))*dt + F*dt
th(i+1) = th(i) + w(i+1)*dt
\end{lstlisting}

\noindent
Esse método utiliza o valor atualizado da velocidade angular para 
calcular a posição, conferindo maior estabilidade numérica, especialmente 
para sistemas oscilatórios.

\subsection*{Construção do mapa de Poincaré}

Durante a simulação, o programa verifica se o argumento da força 
$c_3 i \, dt$ é aproximadamente um múltiplo de $\pi$.  
Essa condição identifica os instantes em que o sistema completa um ciclo 
da força externa.  
Quando isso ocorre, o valor da velocidade angular $\omega$ e do ângulo 
$\theta$ são registrados no arquivo \texttt{saida-2-12694394.txt}, 
formando o conjunto de pontos do mapa de Poincaré:

\begin{lstlisting}
if (abs(mod(rr, pi)) .LE. 1d-3) then
    ival(i+1) = 1
    write(7,8) w(i+1), th(i+1)
else
    ival(i+1) = 0
end if
\end{lstlisting}

\noindent
Esses pontos permitem analisar visualmente a evolução do sistema no 
espaço de fases sob o regime forçado, revelando comportamentos 
periódicos, quase-periódicos ou caóticos.

\subsection*{Armazenamento dos resultados completos}

Além do mapa de Poincaré, o código também grava os dados completos da 
simulação no arquivo \texttt{saida-1-12694394.txt}.  
Antes de salvar, o ângulo é ajustado para permanecer dentro do intervalo 
$[0, 2\pi]$, garantindo a consistência do gráfico de fase.  
O arquivo contém, para cada iteração:

\begin{itemize}
    \item o identificador de ponto de Poincaré (\texttt{ival(i)}),
    \item o tempo ($t = i \cdot \Delta t$),
    \item a velocidade angular ($\omega$),
    \item e o ângulo ($\theta$).
\end{itemize}

\noindent
Os dados são escritos no formato:
\begin{lstlisting}
2 format(I2,3(',',F16.8))
\end{lstlisting}

\noindent
e o arquivo é finalizado com o comando \texttt{close(1)}.

\subsection*{Resumo}

Em resumo, o código implementa a simulação de um pêndulo amortecido e 
forçado, aplicando o método de Euler-Cromer para integração numérica e 
gerando o mapa de Poincaré do sistema.  
A análise dos pontos registrados permite identificar regimes de 
periodicidade, transições para o caos e comportamento determinístico 
complexo, característico de sistemas não lineares forçados.

\section*{Resultados}

\begin{figure}[H]
\centering
\caption{Resultado obtidos para $F_0$ = 1.2}
\centering
\includegraphics[width=16cm]{images/tarefa-D/saida-1-12694394.png}
\caption*{Fonte: Compilado pelo Autor.}
\label{fig:tarefa D - Resultado 1}
\end{figure}

\begin{figure}[H]
\centering
\caption{Resultado obtidos para $F_0$ = 1.2 com filtro de Poincaré}
\centering
\includegraphics[width=16cm]{images/tarefa-D/saida-2-12694394.png}
\caption*{Fonte: Compilado pelo Autor.}
\label{fig:tarefa D - Resultado 2}
\end{figure}

\begin{figure}[H]
\centering
\caption{Resultado obtidos para $F_0$ = 0.5}
\centering
\includegraphics[width=16cm]{images/tarefa-D/saida-3-12694394.png}
\caption*{Fonte: Compilado pelo Autor.}
\label{fig:tarefa D - Resultado 3}
\end{figure}

\begin{figure}[H]
\centering
\caption{Resultado obtidos para $F_0$ = 0.5 com filtro de Poincaré}
\centering
\includegraphics[width=16cm]{images/tarefa-D/saida-4-12694394.png}
\caption*{Fonte: Compilado pelo Autor.}
\label{fig:tarefa D - Resultado 4}
\end{figure}

