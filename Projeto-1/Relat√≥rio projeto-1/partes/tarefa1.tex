\chapter*{Tarefa 1}

\section*{Objetivo}

Nesse problema é pedido para calcular a área e o volume de um torus, o usuário irá receber o raio interno $r_1$ e o raio externo $r_2$ e deverá devolver no terminal o valor da área e do volume.

\section*{Torus}

Em geometria, um torus é uma superfície de revolução gerada ao girar um círculo no espaço tridimensional por uma revolução completa em torno de um eixo que é coplanar com o círculo. Os principais tipos de torus incluem. 


\section*{Calculo}
As formulas utilizadas para calcular a área e o volume do torus são respectivamente Equação-\ref{Eq1-1} e Equação-\ref{Eq1-2}

\begin{equation} \label{Eq1-1}
    A = \int_{0}^{2\pi}Rd\phi\int_{0}^{2\pi}rd\theta = 4\pi R r
\end{equation}

\begin{equation} \label{Eq1-2}
    V = \int_{0}^{2\pi}Rd\phi\int_{0}^{2\pi}d\theta\int_{0}^{r}r'dr' = 2\pi R r^2
\end{equation}
