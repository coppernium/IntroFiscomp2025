\chapter*{Motivação}
O problema do caminhante aleatório é um modelo simples, mas de enorme importância 
na física. Ele descreve o movimento de uma partícula que, a cada passo, escolhe sua 
direção de forma aleatória. Apesar da simplicidade, esse modelo captura a essência 
de processos estocásticos presentes em muitos sistemas naturais e serve como ponto 
de partida para o estudo de fenômenos mais complexos. Além disso, no contexto da 
Física Computacional, a geração de números pseudo-aleatórios desempenha papel 
fundamental, pois permite a implementação eficiente de simulações que reproduzem 
esse tipo de processo probabilístico em computadores.

Na física estatística, o caminhante aleatório está diretamente relacionado à difusão, 
um processo fundamental que explica como partículas se espalham em fluidos e sólidos. 
A equação da difusão e a equação de Fokker-Planck, por exemplo, podem ser derivadas 
a partir desse modelo discreto. Isso mostra como um conceito probabilístico simples 
pode se conectar a leis físicas que regem o transporte de calor, carga elétrica e até 
a propagação de sinais em materiais. Nesse sentido, os objetivos principais dos estudos 
computacionais incluem a implementação de geradores pseudo-aleatórios confiáveis, 
a simulação de caminhantes aleatórios em uma e duas dimensões, e a análise da entropia 
como medida da desordem do sistema.

Além disso, o estudo do caminhante aleatório tem aplicações que vão além da difusão 
clássica, alcançando áreas como mecânica quântica, teoria de polímeros e até sistemas 
biológicos. Ele oferece uma linguagem matemática unificada para descrever flutuações, 
processos de relaxação e até o comportamento coletivo de sistemas complexos. Assim, 
ao unir teoria, simulação computacional e análise estatística, o problema do caminhante 
aleatório revela-se uma ferramenta essencial para compreender e modelar a física em 
múltiplas escalas.
