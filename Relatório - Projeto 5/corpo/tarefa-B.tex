\chapter*{Tarefa - B}

\section*{Código}


% Bloco 1: Função Principal
\begin{lstlisting}[caption={Função principal do código.}, label={lst:main-codigo}]
	program main
	implicit real*8 (a-h,o-z)
	
	call calc()
	call calc2()
	
	end program main
\end{lstlisting}
\vspace*{1\baselineskip}

% Bloco 2: Sub-rotina calc - Parte 1 (Declarações e Constantes)
\begin{lstlisting}[caption={Implementação da subrotina \texttt{calc}: Declarações, constantes e condições iniciais (parte 1 de 3).}, label={lst:calc-part1}]
	subroutine calc()
	implicit real*8 (a-h,o-z)
	parameter (imax=5e3)
	
	dimension x_t(-1:imax), y_t(-1:imax)
	dimension x_j(-1:imax), y_j(-1:imax)
	
	C         Constantes
	
	pi = acos(-1d0)
	dt = 1d0/365d0
	
	GM_S = 4d0*pi*pi
	GM_J = GM_S * 0.0009543d0 * 1    ! multiplica a massa
	GM_T = GM_S * 3.003d-6
	
	C
	GM_S2 = GM_S + GM_J
	
	C         C.I.
	
	x_t(0) = 1d0
	y_t(0) = 0d0
	vx_t  = 0d0
	vy_t  = 2d0*pi
	
	x_j(0) = 5.2d0
	y_j(0) = 0d0
	vx_j  = 0d0
	vy_j  = 2d0*pi*5.2d0/11.86d0
\end{lstlisting}
\vspace*{1\baselineskip}

% Bloco 3: Sub-rotina calc - Parte 2 (Iteração Negativa e Loop Principal)
\begin{lstlisting}[caption={Sub-rotina \texttt{calc}: Interação negativa e loop principal de cálculo (parte 2 de 3).}, label={lst:calc-part2}]
	C         Intereaçao negativa
	x_t(-1) = x_t(0) - vx_t*dt
	y_t(-1) = y_t(0) - vy_t*dt
	
	x_j(-1) = x_j(0) - vx_j*dt
	y_j(-1) = y_j(0) - vy_j*dt
	
	C         Contas
	do i = 0, imax-1
	
	C         Acel. Terra
	r_ts = sqrt(x_t(i)**2 + y_t(i)**2)
	r_tj = sqrt((x_t(i)-x_j(i))**2 + (y_t(i)-y_j(i))**2)
	
	ax_t = -GM_S*x_t(i)/(r_ts**3) - GM_J*(x_t(i)-x_j(i))/(r_tj**3)
	ay_t = -GM_S*y_t(i)/(r_ts**3) - GM_J*(y_t(i)-y_j(i))/(r_tj**3)
	
	C         Acel jupiter
	r_js = sqrt(x_j(i)**2 + y_j(i)**2)
	
	ax_j = -GM_S2*x_j(i)/(r_js**3)
	ay_j = -GM_S2*y_j(i)/(r_js**3)
	
	C         atualiza os dados
	x_t(i+1) = 2d0*x_t(i) - x_t(i-1) + ax_t*dt*dt
	y_t(i+1) = 2d0*y_t(i) - y_t(i-1) + ay_t*dt*dt
	
	x_j(i+1) = 2d0*x_j(i) - x_j(i-1) + ax_j*dt*dt
	y_j(i+1) = 2d0*y_j(i) - y_j(i-1) + ay_j*dt*dt
	
	end do
\end{lstlisting}
\vspace*{1\baselineskip}

% Bloco 4: Sub-rotina calc - Parte 3 (Salvamento de Dados)
\begin{lstlisting}[caption={Sub-rotina \texttt{calc}: Salvamento dos dados calculados em arquivo (parte 3 de 3).}, label={lst:calc-part3}]
	C         Salva
	
	open(unit=1, file='saida_m_1.txt')
	do i = 0, imax
	write(1,2) dt*i, x_t(i), y_t(i), x_j(i), y_j(i)
	end do
	2       format(F16.8, 4(",",F16.8))
	close(1)
	
	end subroutine calc
\end{lstlisting}
\vspace*{1\baselineskip}

% Bloco 5: Sub-rotina calc2 - Parte 1 (Declarações e Constantes)
\begin{lstlisting}[caption={Sub-rotina \texttt{calc2}: Declarações de variáveis e constantes (parte 1 de 4).}, label={lst:calc2-part1}]
	subroutine calc2()
	implicit real*8 (a-h,o-z)
	parameter (imax=1e3)
	
	dimension xj(-1:imax), yj(-1:imax)
	dimension xa1(-1:imax), ya1(-1:imax)
	dimension xa2(-1:imax), ya2(-1:imax)
	dimension xa3(-1:imax), ya3(-1:imax)
	
	C         Constantes
	
	pi    = acos(-1d0)
	dt    = 1d0/365d0
	
	GM_S = 4d0*pi*pi
	GM_J = GM_S * 0.0009543d0
	GM_S2 = GM_S + GM_J !Isso é para jupiter sentir a mudamça de massa
\end{lstlisting}
\vspace*{1\baselineskip}

% Bloco 6: Sub-rotina calc2 - Parte 2 (Condições Iniciais)
\begin{lstlisting}[caption={Sub-rotina \texttt{calc2}: Condições iniciais para Júpiter e asteroides (parte 2 de 4).}, label={lst:calc2-part2}]
	C         Cond Ini
	
	C         Jupiter
	xj(0) = 5.2d0
	yj(0) = 0d0
	vxj   = 0d0
	vyj   = 2d0*pi*5.2d0/11.86d0
	
	C         Asteroide 1
	xa1(0) = 3.0d0
	ya1(0) = 0d0
	vxa1   = 0d0
	vya1   = 3.628d0
	
	C         Asteroide 2
	xa2(0) = 3.276d0
	ya2(0) = 0d0
	vxa2   = 0d0
	vya2   = 3.471d0
	C         Asteroide 3
	xa3(0) = 3.700d0
	ya3(0) = 0d0
	vxa3   = 0d0
	vya3   = 3.267d0
\end{lstlisting}
\vspace*{1\baselineskip}

% Bloco 7: Sub-rotina calc2 - Parte 3 (Setup Verlet e Loop Principal)
\begin{lstlisting}[caption={Sub-rotina \texttt{calc2}: Setup do método de Verlet e início do loop principal (parte 3 de 4).}, label={lst:calc2-part3}]
	C         Setup do i-1 do verlat
	
	xj(-1) = xj(0) - vxj*dt
	yj(-1) = yj(0) - vyj*dt
	
	xa1(-1) = xa1(0) - vxa1*dt
	ya1(-1) = ya1(0) - vya1*dt
	
	xa2(-1) = xa2(0) - vxa2*dt
	ya2(-1) = ya2(0) - vya2*dt
	
	xa3(-1) = xa3(0) - vxa3*dt
	ya3(-1) = ya3(0) - vya3*dt
	
	C         Realiza as contas
	do i = 0, imax-1
	
	C         Jupiter
	rjs = sqrt(xj(i)**2 + yj(i)**2)
	axj = -GM_S2*xj(i)/(rjs**3)
	ayj = -GM_S2*yj(i)/(rjs**3)
	
	C         teroide 1
	ras = sqrt(xa1(i)**2 + ya1(i)**2)
	raj = sqrt((xa1(i)-xj(i))**2 + (ya1(i)-yj(i))**2)
	
	axa1 = -GM_S*xa1(i)/(ras**3) - GM_J*(xa1(i)-xj(i))/(raj**3)
	aya1 = -GM_S*ya1(i)/(ras**3) - GM_J*(ya1(i)-yj(i))/(raj**3)
	
	C         teroide2
	ras = sqrt(xa2(i)**2 + ya2(i)**2)
	raj = sqrt((xa2(i)-xj(i))**2 + (ya2(i)-yj(i))**2)
	
	axa2 = -GM_S*xa2(i)/(ras**3) - GM_J*(xa2(i)-xj(i))/(raj**3)
	aya2 = -GM_S*ya2(i)/(ras**3) - GM_J*(ya2(i)-yj(i))/(raj**3)
	
	C         teroide 3
	ras = sqrt(xa3(i)**2 + ya3(i)**2)
	raj = sqrt((xa3(i)-xj(i))**2 + (ya3(i)-yj(i))**2)
	
	axa3 = -GM_S*xa3(i)/(ras**3) - GM_J*(xa3(i)-xj(i))/(raj**3)
	aya3 = -GM_S*ya3(i)/(ras**3) - GM_J*(ya3(i)-yj(i))/(raj**3)
	
	C         atualizaoção das posições usando verlat
	xj(i+1) = 2*xj(i) - xj(i-1) + axj*dt*dt
	yj(i+1) = 2*yj(i) - yj(i-1) + ayj*dt*dt
	
	xa1(i+1) = 2*xa1(i) - xa1(i-1) + axa1*dt*dt
	ya1(i+1) = 2*ya1(i) - ya1(i-1) + aya1*dt*dt
	
	xa2(i+1) = 2*xa2(i) - xa2(i-1) + axa2*dt*dt
	ya2(i+1) = 2*ya2(i) - ya2(i-1) + aya2*dt*dt
	
	xa3(i+1) = 2*xa3(i) - xa3(i-1) + axa3*dt*dt
	ya3(i+1) = 2*ya3(i) - ya3(i-1) + aya3*dt*dt
	
	end do
\end{lstlisting}
\vspace*{1\baselineskip}

% Bloco 8: Sub-rotina calc2 - Parte 4 (Salvamento de Dados e Fim)
\begin{lstlisting}[caption={Sub-rotina \texttt{calc2}: Salvamento dos dados calculados em arquivos e finalização (parte 4 de 4).}, label={lst:calc2-part4}]
	C         Salva
	
	open(unit=1,file='asteroides_saida.txt')
	open(unit=2,file='asteroides_saida_2.txt')
	
	do i = 0, imax
	write(1,10) dt*i, xj(i), yj(i), xa1(i), ya1(i)
	write(2,10) dt*i, xa2(i), ya2(i), xa3(i), ya3(i)
	end do
	
	10      format(F12.6,4(",",F12.6))
	
	close(1)
	close(2)
	
	end subroutine calc2
\end{lstlisting}

 \section*{Descrição do Código}

O programa \texttt{main} tem como objetivo resolver numericamente o problema de três corpos
(Terra, Sol e Júpiter) e, em seguida, estudar as perturbações gravitacionais de Júpiter sobre
três asteroides localizados na região do cinturão de asteroides. Para isso, são utilizadas duas
subrotinas independentes:

\begin{lstlisting}
	call calc()
	call calc2()
\end{lstlisting}

Cada subrotina usa o método de Verlet para integrar as equações de movimento e grava os
resultados em arquivos para posterior análise gráfica.

\subsection*{Subrotina \texttt{calc}: Terra--Sol--Júpiter}

A primeira subrotina implementa o problema de três corpos envolvendo o Sol (fixo na origem), 
a Terra e Júpiter. O número máximo de iterações utilizado é:

\begin{lstlisting}
	parameter (imax = 5e3)
\end{lstlisting}

Foram definidos quatro vetores para armazenar as posições da Terra e de Júpiter:

\begin{center}
	\texttt{x\_t, y\_t, x\_j, y\_j}
\end{center}

todos indexados de \(-1\) até \(\texttt{imax}\), como requerido pelo método de Verlet.

\subsubsection*{Constantes físicas}

Os parâmetros usados são definidos como:

\begin{lstlisting}
	pi = acos(-1d0)
	dt = 1d0/365d0
	GM_S = 4*pi*pi
	GM_J = GM_S * 0.0009543d0
	GM_T = GM_S * 3.003d-6
	GM_S_eff = GM_S + GM_J
\end{lstlisting}

onde:

\begin{itemize}
	\item \(GM_S = 4\pi^2\) é a constante gravitacional do Sol em unidades astronômicas,
	\item \(GM_J\) é o valor ajustado para a massa de Júpiter,
	\item \(GM_S^\text{eff}\) representa a massa efetiva sentida por Júpiter devido ao Sol fixo.
\end{itemize}

\subsubsection*{Condições iniciais}

As condições iniciais da Terra e de Júpiter são:

\begin{align}
	x_T(0) &= 1, & y_T(0) &= 0, &
	v_{x,T} &= 0, & v_{y,T} &= 2\pi, \\
	x_J(0) &= 5.2, & y_J(0) &= 0, &
	v_{x,J} &= 0, &
	v_{y,J} &= 2\pi\frac{5.2}{11.86}.
\end{align}

\noindent
As acelerações iniciais incluem interações Terra–Sol e Terra–Júpiter:

\begin{align}
	\vec{a}_T = 
	-\,\frac{GM_S}{r_{TS}^3}\vec{r}_{TS}
	-\,\frac{GM_J}{r_{TJ}^3}\vec{r}_{TJ},
\end{align}

e, para Júpiter (com Sol fixo e massa efetiva):

\begin{align}
	\vec{a}_J = 
	-\,\frac{GM_S^\text{eff}}{r_{JS}^3}\vec{r}_{JS}.
\end{align}

\subsubsection*{Passo inicial do método de Verlet}

O método exige as posições no tempo \(-\Delta t\), obtidas por:

\begin{align}
	x(-1) = x(0) - v_x\,dt + \frac{1}{2}a_x\,dt^2, \qquad
	y(-1) = y(0) - v_y\,dt + \frac{1}{2}a_y\,dt^2.
\end{align}

\subsubsection*{Integração numérica}

A integração é realizada pelo método de Verlet:

\begin{align}
	x_{i+1} &= 2x_i - x_{i-1} + a_x\,dt^2, \\
	y_{i+1} &= 2y_i - y_{i-1} + a_y\,dt^2.
\end{align}

As acelerações são recalculadas a cada passo considerando as forças:

\begin{itemize}
	\item Terra–Sol,
	\item Terra–Júpiter,
	\item Júpiter–Sol (massa efetiva).
\end{itemize}

O laço principal avança ambas as órbitas simultaneamente.

\subsubsection*{Saída de dados}

As coordenadas são gravadas no arquivo:

\begin{center}
	\texttt{saida\_m\_1.txt}
\end{center}

no formato:

\[
t,\; x_T(t),\; y_T(t),\; x_J(t),\; y_J(t).
\]

Isso permite analisar graficamente a órbita terrestre sob perturbações de Júpiter,
atendendo aos itens \textbf{b1} e \textbf{b2} da tarefa.

\subsection*{Subrotina \texttt{calc2}: Asteroides sob a perturbação de Júpiter}

A segunda subrotina implementa o item \textbf{b3} da tarefa, estudando a influência 
de Júpiter sobre três asteroides no cinturão principal.

O número de iterações é:

\begin{lstlisting}
	parameter (imax = 1000)
\end{lstlisting}

São definidos:

\begin{itemize}
	\item posição de Júpiter: \texttt{xj, yj},
	\item três asteroides: \texttt{xa1, ya1}, \texttt{xa2, ya2}, \texttt{xa3, ya3}.
\end{itemize}

\subsubsection*{Condições iniciais}

As condições de Júpiter são:

\begin{align}
	x_J(0) &= 5.2, &
	y_J(0) &= 0, &
	v_{y,J} &= 2.755.
\end{align}

Os asteroides são inicializados conforme os dados do enunciado:

\begin{align}
	(x_{A1}, y_{A1}, v_{y,A1}) &= (3.0,\; 0,\; 3.628), \\
	(x_{A2}, y_{A2}, v_{y,A2}) &= (3.276,\; 0,\; 3.471), \\
	(x_{A3}, y_{A3}, v_{y,A3}) &= (3.700,\; 0,\; 3.267).
\end{align}

\subsubsection*{Dinâmica gravitacional}

Cada asteroide sofre duas forças:

\begin{align}
	\vec{F} = 
	-\,\frac{GM_S}{r_{AS}^3}\vec{r}_{AS}
	-\,\frac{GM_J}{r_{AJ}^3}\vec{r}_{AJ}.
\end{align}

O efeito dos asteroides sobre Júpiter é desprezado, conforme pedido.

\subsubsection*{Integração}

O mesmo esquema de Verlet é utilizado:

\begin{align}
	x_{i+1} &= 2x_i - x_{i-1} + a_x\,dt^2, \\
	y_{i+1} &= 2y_i - y_{i-1} + a_y\,dt^2.
\end{align}

Todos os quatro corpos (Júpiter + 3 asteroides) são atualizados em cada iteração.

\subsubsection*{Saída de dados}

Dois arquivos são produzidos:

\begin{itemize}
	\item \texttt{asteroides\_saida.txt}: Júpiter + Asteroide I.
	\item \texttt{asteroides\_saida\_2.txt}: Asteroides II e III.
\end{itemize}

O formato de saída é:

\[
t,\; x(t),\; y(t),\; x'(t),\; y'(t).
\]

Esses dados permitem identificar ressonâncias orbitais e lacunas de Kirkwood, conforme a tarefa solicita.

\section*{Resumo}

O código implementa:

\begin{itemize}
	\item o método de Verlet para todas as partículas;
	\item a interação Terra–Sol–Júpiter (problema de três corpos);
	\item a perturbação de Júpiter sobre três asteroides do cinturão principal;
	\item saídas organizadas para análise posterior das órbitas.
\end{itemize}

O programa atende completamente aos itens \textbf{b1}, \textbf{b2} e \textbf{b3} da Tarefa B do projeto.



\section*{Resultados }

\begin{figure}[H]
	\centering
	\caption{Exemplo de órbita de um planeta.}
	\centering
	\includegraphics[width=16cm]{images/tarefa-B/fig3.png}
	\caption*{Fonte: Compilado pelo Autor.}
	\label{fig:tarefa B - Resultado 1}
\end{figure}

\begin{figure}[H]
	\centering
	\caption{Exemplo de órbita de um planeta.}
	\centering
	\includegraphics[width=16cm]{images/tarefa-B/fig_m1.png}
	\caption*{Fonte: Compilado pelo Autor.}
	\label{fig:tarefa B - Resultado 2}
\end{figure}

\begin{figure}[H]
	\centering
	\caption{Exemplo de órbita de um planeta.}
	\centering
	\includegraphics[width=16cm]{images/tarefa-B/fig_m100.png}
	\caption*{Fonte: Compilado pelo Autor.}
	\label{fig:tarefa B - Resultado 3}
\end{figure}

\begin{figure}[H]
	\centering
	\caption{Exemplo de órbita de um planeta.}
	\centering
	\includegraphics[width=16cm]{images/tarefa-B/fig_m1000.png}
	\caption*{Fonte: Compilado pelo Autor.}
	\label{fig:tarefa B - Resultado 4}
\end{figure}

\begin{figure}[H]
	\centering
	\caption{Exemplo de órbita de um planeta.}
	\centering
	\includegraphics[width=16cm]{images/tarefa-B/fig_m10000.png}
	\caption*{Fonte: Compilado pelo Autor.}
	\label{fig:tarefa B - Resultado 5}
\end{figure}

\begin{figure}[H]
	\centering
	\caption{Exemplo de órbita de um planeta.}
	\centering
	\includegraphics[width=16cm]{images/tarefa-B/fig2.png}
	\caption*{Fonte: Compilado pelo Autor.}
	\label{fig:tarefa B - Resultado 6}
\end{figure}

\begin{figure}[H]
	\centering
	\caption{Lacunas de Kirkwood.}
	\centering
	\includegraphics[width=16cm]{images/tarefa-B/Kirkwood-20060509.png}
	\caption*{Fonte: Retirado da Wikipédia.}
	\label{fig:tarefa B - Resultado 7}
\end{figure}

\section*{Discussão dos Resultados}

Nesta simulação computacional analisamos o movimento de três asteroides na região do Cinturão Principal, levando em conta apenas as interações gravitacionais com o Sol e com Júpiter. A presença de Júpiter é particularmente importante, pois o planeta possui massa suficientemente grande para perturbar órbitas próximas e modificar lentamente seus semieixos maiores, excentricidades e ângulos orbitais.

Os três asteroides foram escolhidos com distâncias iniciais de 3{,}0 UA, 3{,}276 UA e 3{,}7 UA, valores próximos de regiões reais do cinturão. Os resultados mostram claramente que:

\begin{itemize}
	\item O asteroide mais externo (3{,}7 UA) mantém uma órbita aproximadamente estável e quase circular ao longo do tempo.
	\item O asteroide intermediário (3{,}276 UA) apresenta perturbações moderadas, mas ainda preserva uma órbita aproximadamente regular.
	\item O asteroide interno (3{,}0 UA), por outro lado, sofre variações mais fortes em sua órbita, especialmente na longitude, com pequenas oscilações que se tornam mais evidentes quando Júpiter passa próximo em sua órbita.
\end{itemize}

Essas diferenças de comportamento estão diretamente relacionadas ao fenômeno conhecido como \textbf{ressonâncias orbitais}, fundamentais para compreender a estrutura do cinturão de asteroides.

\subsection*{Lacunas de Kirkwood}

As \textbf{Lacunas de Kirkwood} são regiões do cinturão de asteroides onde praticamente não existem objetos. Elas foram descobertas pelo astrônomo Daniel Kirkwood no século XIX e correspondem a posições onde os asteroides entrariam em ressonância de período com Júpiter.

Uma ressonância ocorre quando:

\[
\frac{T_{\text{ast}}}{T_J} = \frac{q}{p},
\]

com $p$ e $q$ inteiros pequenos.

\noindent Alguns exemplos reais de ressonâncias e suas posições no cinturão:

\begin{center}
	\begin{tabular}{ccc}
		\hline
		\textbf{Ressonância} & \textbf{Distância (UA)} & \textbf{Efeito} \\
		\hline
		3:1 & 2.50 & Lacuna profunda \\
		5:2 & 2.82 & Lacuna \\
		7:3 & 2.96 & Lacuna \\
		2:1 & 3.27 & Grande lacuna \\
		\hline
	\end{tabular}
\end{center}

Quando um asteroide entra em uma destas ressonâncias:

\begin{itemize}
	\item recebe pequenos “empurrões” gravitacionais repetidos de Júpiter,
	\item sua excentricidade cresce lentamente,
	\item sua órbita se torna instável,
	\item e eventualmente ele é ejetado daquela região.
\end{itemize}

\subsection*{Interpretação dos resultados obtidos}

Na simulação realizada, observamos que:

\begin{itemize}
	\item O asteroide em \textbf{3.0 UA} está muito próximo da ressonância 7:3, por isso sofre perturbações mais intensas.
	\item O asteroide em \textbf{3.276 UA} está praticamente na grande lacuna associada à ressonância 2:1, o que explica possíveis instabilidades mais fortes se a simulação for estendida por mais tempo.
	\item O asteroide em \textbf{3.7 UA}, longe das principais ressonâncias, apresenta um movimento claramente mais estável ao longo da integração.
\end{itemize}

Assim, o comportamento observado nos gráficos e na animação está em total acordo com o que é previsto pela dinâmica orbital real e com o padrão das Lacunas de Kirkwood no cinturão principal. O estudo evidencia como perturbações gravitacionais sutis, mas repetidas ao longo de milhares de períodos orbitais, moldam a estrutura do cinturão e explicam por que certas regiões permanecem praticamente vazias enquanto outras mantêm grande população de corpos.


