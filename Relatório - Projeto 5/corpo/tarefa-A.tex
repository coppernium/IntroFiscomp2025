\chapter*{Tarefa - A}

\section*{Código}


\vspace*{1\baselineskip}

\begin{figure}[H]
\centering
\caption{Função principal do código.}
\centering

\begin{lstlisting}
	     program main
		implicit real*8 (a-h,o-z)
		a = 39.53d0
		call calc(a)
		end program main
\end{lstlisting}

\caption*{Fonte: Compilado pelo Autor.}
\label{fig:tarefa 1 - função principal do código}
\end{figure}

\vspace*{1\baselineskip}

\begin{figure}[H]
	\centering
	\caption{Subrotina que realiza os cálculos, para um certo valor de a.}
	\centering
	
	\begin{lstlisting}

subroutine calc(a)
parameter (imax=1e5)
implicit real*8 (a-h,o-z)
dimension x(-1:imax),y(-1:imax)

C   Constantes
pi = acos(-1d0)
dt = 10d0/365d0
GM = 4*pi*pi

C   Val in
x(0) = 1d0*a !Distancia em UA
y(0) = 0d0
vx = 0d0
vy = 2.0d0*pi/sqrt(a)


C  x(-1) e y(-1)
x(-1) = x(0) - vx*dt
y(-1) = y(0) - vy*dt 

C       Salva para lei de Kepler
io = 0d0 !Variavel de segurança para pegar apenas a primeira volta

	\end{lstlisting}
	
	\caption*{Fonte: Compilado pelo Autor.}
	\label{fig:tarefa 1 - Subrotina}
\end{figure}

\vspace*{1\baselineskip}


\begin{figure}[H]
	\centering
	\caption{Subrotina que realiza os cálculos, para um certo valor de a.}
	\centering
	
	\begin{lstlisting}
		C Calc
		do i = 0,imax-1
		r = sqrt((x(i)**2) + (y(i)**2))
		
		ax = - GM*x(i)/(r**3)
		ay = - GM*y(i)/(r**3)
		
		x(i+1) = 2d0*x(i) - x(i-1) + ax*dt*dt
		y(i+1) = 2d0*y(i) - y(i-1) + ay*dt*dt
		
		theta_new = atan2(y(i+1), x(i+1))
		if ((theta_old .LT. 0d0) .and. (theta_new .GE. 0d0)) then
		if (io .GT. 0d0) then
		goto 7
		end if
		
		if (i*dt .GT. 5d0*dt) then      ! para n detectar o comeco
		periodo = i * dt
		write(*,3) periodo, r, (periodo**2)/(r**3)
		3                       format(F12.4,2(",",F12.4))
		end if
		io = 1d0
		end if
		
		theta_old = theta_new
		7       continue
		end do
		
		C   Salva
		open(unit=1,file='saida-2-12694394.txt')
		do i = 0,imax
		
		write(1,2) dt*i,x(i),y(i)
		
		end do
		2       format(F16.8,2(",",F16.8))
		close(1)
		
		
		end subroutine calc
	\end{lstlisting}
	
	\caption*{Fonte: Compilado pelo Autor.}
	\label{fig:tarefa 1 - Subrotina parte 2}
\end{figure}

\vspace*{1\baselineskip}
\newpage

\section*{Descrição do código}

O programa \texttt{main} tem como finalidade calcular numericamente a 
órbita de um corpo sob atração gravitacional central, utilizando um 
integrador de segunda ordem para resolver as equações do movimento.  

\noindent
Ele utiliza o comando:

\begin{lstlisting}
	implicit real*8 (a-h,o-z)
\end{lstlisting}

\noindent
o que define como números reais de dupla precisão todas as variáveis cujos 
nomes se iniciam com as letras \textbf{a}--\textbf{h} e \textbf{o}--\textbf{z}.  

\noindent
Em seguida, o programa principal define o parâmetro:

\begin{lstlisting}
	a = 1d0
\end{lstlisting}

\noindent
que representa o raio inicial da órbita em unidades astronômicas (UA). 
Por fim, chama a subrotina responsável pelos cálculos:

\begin{lstlisting}
	call calc(a)
\end{lstlisting}

\subsection*{Subrotina \texttt{calc}}

A subrotina \texttt{calc} recebe o parâmetro \texttt{a} e define o número 
máximo de iterações como:

\begin{lstlisting}
	parameter (imax = 1e4)
\end{lstlisting}

\noindent
Os vetores \texttt{x(-1:imax)} e \texttt{y(-1:imax)} armazenam as posições 
cartesianas do corpo ao longo da integração.

\vspace{1\baselineskip}
\noindent
As constantes físicas utilizadas são:

\begin{lstlisting}
	pi = acos(-1d0)
	dt = 1d0/365d0
	GM = 4*pi*pi
	ec = 0.5d0
\end{lstlisting}

\noindent
O passo temporal \texttt{dt} corresponde a um dia em unidades de ano, e 
\texttt{GM} é expresso de modo que sistemas keplerianos possam ser simulados 
em unidades astronômicas sem necessidade de conversões adicionais.  
O parâmetro \texttt{ec} representa o fator de controle da velocidade inicial, 
permitindo estudar órbitas com diferentes excentricidades.

\subsection*{Condições iniciais}

As condições iniciais de posição são definidas como:

\begin{equation}
	x(0) = a, \qquad y(0) = 0
\end{equation}

\noindent
e a velocidade inicial é exclusivamente tangencial:

\begin{equation}
	v_x = 0, \qquad v_y = ec \,\frac{2\pi}{\sqrt{a}}
\end{equation}

\noindent
Além disso, como o método numérico utilizado requer um passo anterior, os 
valores de \texttt{x(-1)} e \texttt{y(-1)} são estimados por:

\begin{lstlisting}
	x(-1) = x(0) - vx*dt
	y(-1) = y(0) - vy*dt
\end{lstlisting}

\noindent
Esses valores representam uma aproximação da posição um passo antes do 
instante inicial.

\subsection*{Método numérico}

A integração é realizada pelo método de  Verlet, dado por:

\begin{equation}
	x_{i+1} = 2x_i - x_{i-1} + a_x\, dt^2,
	\qquad
	y_{i+1} = 2y_i - y_{i-1} + a_y\, dt^2
\end{equation}

\noindent
com as acelerações calculadas pela lei da gravitação universal:

\begin{equation}
	a_x = -\frac{GM x}{r^3}, \qquad
	a_y = -\frac{GM y}{r^3}, \qquad
	r = \sqrt{x^2 + y^2}
\end{equation}


\subsection*{Cálculo da área varrida}

Em cada passo da simulação, a área varrida pelo vetor de posição é 
aproximada pela fórmula:

\begin{equation}
	A_i = \frac{1}{2} \left| x_i\,y_{i+1} - x_{i+1}\,y_i \right|
\end{equation}

\noindent
e é registrada no arquivo:

\begin{center}
	\texttt{saida-1-12694394.txt}
\end{center}

\noindent
Esse cálculo permite verificar a segunda lei de Kepler, que estabelece que 
a área varrida por unidade de tempo permanece constante.

\subsection*{Cálculo do período orbital}

Para determinar o período, o código monitora o ângulo polar:

\begin{equation}
	\theta = \mathrm{atan2}(y, x)
\end{equation}

\noindent
e identifica quando o corpo cruza novamente o eixo \(x\) no semi-eixo 
positivo. A condição usada é:

\begin{equation}
	\theta_{\text{old}} < 0
	\quad \text{e} \quad
	\theta_{\text{new}} \ge 0
\end{equation}

Após ignorar as primeiras iterações, que poderiam detectar o instante 
inicial, o período é calculado por:

\begin{equation}
	T = i\,dt
\end{equation}

\noindent
Além disso, são impressos o raio e a razão:

\begin{equation}
	\frac{T^2}{r^3}
\end{equation}

permitindo verificar numericamente a terceira lei de Kepler.

\subsection*{Gravação da trajetória}

Ao final da simulação, os valores de tempo e posição são armazenados em:

\begin{center}
	\texttt{saida-2-12694394.txt}
\end{center}

\noindent
no formato:

\begin{equation}
	t,\; x(t),\; y(t)
\end{equation}

\subsection*{Resumo}

O código implementa:

\begin{itemize}
	\item um integrador de segunda ordem do tipo Verlet;
	\item condições iniciais ajustáveis via parâmetro \texttt{a};
	\item determinação automática do período orbital;
	\item verificação da terceira lei de Kepler;
	\item armazenamento da trajetória completa em arquivo.
\end{itemize}

\noindent
Trata-se de uma implementação eficiente para o estudo de órbitas keplerianas 
e propriedades fundamentais do movimento sob gravitação central.

\section*{Resultados}

\begin{table}[h!]
	\centering
	\caption{Período orbital, raio e razão $T^2/a^3$ dos planetas.}
	\label{table:kepler}
	\begin{NiceTabular}{c|ccc}[hvlines, columns-width=3cm]
		\CodeBefore
		\rowcolor{cyan}{1}
		\rowcolors{2}{cyan!25}{cyan!15}
		\Body
		\RowStyle[color=white, bold]{}
		Planeta & Período (anos) & Raio (UA) & $T^2/a^3$ \\
		Mercúrio & 0.2436 & 0.3900 & 1.0001 \\
		Vênus & 0.6107 & 0.7200 & 0.9992 \\
		Terra & 1.0000 & 1.0000 & 1.0000 \\
		Marte & 1.8740 & 1.5200 & 1.0000 \\
		Júpiter & 11.8578 & 5.2000 & 1.0000 \\
		Saturno & 28.0871 & 9.2400 & 1.0000 \\
		Urano & 84.0548 & 19.1900 & 0.9998 \\
		Netuno & 164.7945 & 30.0600 & 0.9998 \\
		Plutão & 248.5205 & 39.5300 & 0.9999 \\
	\end{NiceTabular}
	\vspace*{1\baselineskip}
	\caption*{Fonte: Compilado pelo Autor}
\end{table}


\begin{figure}[H]
	\centering
	\caption{Área varrida por um planeta entre as interações $i$ e $i+1$.}
	\centering
	\includegraphics[width=16cm]{images/tarefa-A/fig1.png}
	\caption*{Fonte: Compilado pelo Autor.}
	\label{fig:tarefa A - Resultado 2}
\end{figure}

\begin{figure}[H]
	\centering
	\caption{Exemplo de órbita de um planeta.}
	\centering
	\includegraphics[width=16cm]{images/tarefa-A/fig2.png}
	\caption*{Fonte: Compilado pelo Autor.}
	\label{fig:tarefa A - Resultado 3}
\end{figure}

\section*{Discussão dos resultados}

