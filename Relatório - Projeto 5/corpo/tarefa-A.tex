\chapter*{Tarefa - A}

\section*{Código}


\vspace*{1\baselineskip}

\begin{figure}[H]
\centering
\caption{Função principal do código.}
\centering

\begin{lstlisting}
	     program main
		implicit real*8 (a-h,o-z)
		a = 39.53d0
		call calc(a)
		end program main
\end{lstlisting}

\caption*{Fonte: Compilado pelo Autor.}
\label{fig:tarefa 1 - função principal do código}
\end{figure}

\vspace*{1\baselineskip}

\begin{figure}[H]
	\centering
	\caption{Subrotina que realiza os cálculos, para um certo valor de a.}
	\centering
	
	\begin{lstlisting}

subroutine calc(a)
parameter (imax=1e5)
implicit real*8 (a-h,o-z)
dimension x(-1:imax),y(-1:imax)

C   Constantes
pi = acos(-1d0)
dt = 10d0/365d0
GM = 4*pi*pi

C   Val in
x(0) = 1d0*a !Distancia em UA
y(0) = 0d0
vx = 0d0
vy = 2.0d0*pi/sqrt(a)


C  x(-1) e y(-1)
x(-1) = x(0) - vx*dt
y(-1) = y(0) - vy*dt 

C       Salva para lei de Kepler
io = 0d0 !Variavel de segurança para pegar apenas a primeira volta

	\end{lstlisting}
	
	\caption*{Fonte: Compilado pelo Autor.}
	\label{fig:tarefa 1 - Subrotina}
\end{figure}

\vspace*{1\baselineskip}


\begin{figure}[H]
	\centering
	\caption{Subrotina que realiza os cálculos, para um certo valor de a.}
	\centering
	
	\begin{lstlisting}
		C Calc
		do i = 0,imax-1
		r = sqrt((x(i)**2) + (y(i)**2))
		
		ax = - GM*x(i)/(r**3)
		ay = - GM*y(i)/(r**3)
		
		x(i+1) = 2d0*x(i) - x(i-1) + ax*dt*dt
		y(i+1) = 2d0*y(i) - y(i-1) + ay*dt*dt
		
		theta_new = atan2(y(i+1), x(i+1))
		if ((theta_old .LT. 0d0) .and. (theta_new .GE. 0d0)) then
		if (io .GT. 0d0) then
		goto 7
		end if
		
		if (i*dt .GT. 5d0*dt) then      ! para n detectar o comeco
		periodo = i * dt
		write(*,3) periodo, r, (periodo**2)/(r**3)
		3                       format(F12.4,2(",",F12.4))
		end if
		io = 1d0
		end if
		
		theta_old = theta_new
		7       continue
		end do
		
		C   Salva
		open(unit=1,file='saida-2-12694394.txt')
		do i = 0,imax
		
		write(1,2) dt*i,x(i),y(i)
		
		end do
		2       format(F16.8,2(",",F16.8))
		close(1)
		
		
		end subroutine calc
	\end{lstlisting}
	
	\caption*{Fonte: Compilado pelo Autor.}
	\label{fig:tarefa 1 - Subrotina parte 2}
\end{figure}

\vspace*{1\baselineskip}
\newpage
\section*{Descrição do código}

O programa \texttt{main} tem como finalidade calcular numericamente a 
órbita de um corpo sob atração gravitacional central, utilizando um método 
de diferenças finitas de segunda ordem para integrar as equações do movimento.  

\noindent
Inicialmente, o programa utiliza o comando:

\begin{lstlisting}
	implicit real*8 (a-h,o-z)
\end{lstlisting}

\noindent
que define todas as variáveis cujos nomes começam com as letras de 
\textbf{a} a \textbf{h} e \textbf{o} a \textbf{z} como números reais de 
dupla precisão.  

\noindent
Em seguida, o programa principal define o parâmetro:

\begin{lstlisting}
	a = 39.53d0
\end{lstlisting}

\noindent
que representa a distância inicial em unidades astronômicas (UA), podendo 
ser alterada conforme o sistema orbital analisado. Por fim, o programa chama 
a subrotina responsável pelos cálculos:

\begin{lstlisting}
	call calc(a)
\end{lstlisting}

\subsection*{Subrotina \texttt{calc}}

A subrotina \texttt{calc} recebe o parâmetro \texttt{a} e define o número 
máximo de iterações como:

\begin{lstlisting}
	parameter (imax = 1e5)
\end{lstlisting}

\noindent
Os vetores \texttt{x(-1:imax)} e \texttt{y(-1:imax)} armazenam as posições 
cartesianas do corpo em cada passo da simulação.

\vspace{1\baselineskip}
\noindent
As constantes físicas e parâmetros numéricos são definidos como:

\begin{lstlisting}
	pi = acos(-1d0)
	dt = 10d0/365d0
	GM = 4*pi*pi
\end{lstlisting}

\noindent
onde \texttt{dt} representa um passo temporal equivalente a 10 dias, e 
\texttt{GM} é escrito em unidades astronômicas e anos, permitindo o uso de 
órbitas keplerianas sem necessidade de conversão adicional.

\subsection*{Condições iniciais}

As condições iniciais de posição são definidas por:

\[
x(0) = a, \qquad y(0) = 0,
\]

\noindent
enquanto a velocidade inicial é puramente tangencial:

\[
v_x = 0, \qquad 
v_y = \frac{2\pi}{\sqrt{a}}.
\]

\noindent
Para o método numérico utilizado, é necessário especificar também os valores 
de \texttt{x(-1)} e \texttt{y(-1)}, estimados como:

\begin{lstlisting}
	x(-1) = x(0) - vx*dt
	y(-1) = y(0) - vy*dt
\end{lstlisting}

\noindent
Esses valores simulam o instante anterior ao início da integração, 
permitindo o uso do método de segunda ordem.

\subsection*{Método numérico}

A subrotina emprega um método de integração do tipo Verlet, dado por:

\[
x_{i+1} = 2x_i - x_{i-1} + a_x\, dt^2,
\qquad
y_{i+1} = 2y_i - y_{i-1} + a_y\, dt^2,
\]

\noindent
onde as acelerações são calculadas pela lei da gravitação universal:

\[
a_x = -\frac{GM\,x}{r^3}, \qquad
a_y = -\frac{GM\,y}{r^3},
\qquad
r = \sqrt{x^2 + y^2}.
\]

\noindent
Esse método é adequado para sistemas orbitais, pois apresenta boa 
conservação da energia mecânica.

\subsection*{Cálculo do período orbital}

O código realiza o monitoramento do ângulo polar:

\[
\theta = \mathrm{atan2}(y, x)
\]

\noindent
para identificar quando o corpo cruza novamente o eixo positivo do eixo \(x\).  
A condição:

\[
\theta_{\text{old}} < 0 \quad \text{e} \quad \theta_{\text{new}} \ge 0
\]

\noindent
indica uma passagem completa. Após um intervalo de segurança para evitar a 
detecção do próprio início do movimento, o período orbital é calculado como:

\[
T = i \cdot dt.
\]

\noindent
O programa também imprime o valor de:

\[
\frac{T^2}{r^3},
\]

\noindent
permitindo verificar numericamente a terceira lei de Kepler.

\subsection*{Gravação dos resultados}

Ao final da simulação, os valores de tempo e posição são gravados no arquivo:

\begin{center}
	\texttt{saida-2-12694394.txt}
\end{center}

\noindent
no formato:

\[
t,\; x(t),\; y(t).
\]

\noindent

\subsection*{Resumo}

O código implementa:

\begin{itemize}
	\item um integrador de segunda ordem do tipo Verlet;
	\item condições iniciais ajustáveis via parâmetro \texttt{a};
	\item determinação automática do período orbital;
	\item verificação da terceira lei de Kepler;
	\item armazenamento da trajetória completa em arquivo.
\end{itemize}

\noindent
Trata-se de uma implementação eficiente para o estudo de órbitas keplerianas 
e propriedades fundamentais do movimento sob gravitação central.

\subsection*{Resumo}

Em síntese, o código compara numericamente os métodos de Euler e 
Euler-Cromer na simulação de um pêndulo simples, permitindo observar 
diferenças na conservação da energia ao longo do tempo.  
Cada método gera dois arquivos de saída: um contendo os parâmetros 
dinâmicos ($\omega$ e $\theta$) e outro contendo as energias calculadas em 
cada instante.


\section*{Resultados}


\begin{table}[h!]
	\centering
	\caption{Período orbital, raio e razão $T^2/a^3$ dos planetas.}
	\label{table:kepler}
	\begin{NiceTabular}{c|ccc}[hvlines, columns-width=3cm]
		\CodeBefore
		\rowcolor{cyan}{1}
		\rowcolors{2}{cyan!25}{cyan!15}
		\Body
		\RowStyle[color=white, bold]{}
		Planeta & Período (anos) & Raio (UA) & $T^2/a^3$ \\
		Mercúrio & 0.2436 & 0.3900 & 1.0001 \\
		Vênus & 0.6107 & 0.7200 & 0.9992 \\
		Terra & 1.0000 & 1.0000 & 1.0000 \\
		Marte & 1.8740 & 1.5200 & 1.0000 \\
		Júpiter & 11.8578 & 5.2000 & 1.0000 \\
		Saturno & 28.0871 & 9.2400 & 1.0000 \\
		Urano & 84.0548 & 19.1900 & 0.9998 \\
		Netuno & 164.7945 & 30.0600 & 0.9998 \\
		Plutão & 248.5205 & 39.5300 & 0.9999 \\
	\end{NiceTabular}
	\vspace*{1\baselineskip}
	\caption*{Fonte: Compilado pelo Autor}
\end{table}
