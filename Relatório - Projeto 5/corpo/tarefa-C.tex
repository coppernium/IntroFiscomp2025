\chapter*{Tarefa - C}

\section*{Enunciado}

\begin{figure}[H]
\centering
\caption{Enunciado da Tarefa C.}
\centering
\includegraphics[width=16cm]{images/tarefa-C/enunciado-tarefaC.png}
\caption*{Fonte: Compilado pelo Autor.}
\label{fig:tarefa A - Enunciado}
\end{figure}


\section*{Código}


\vspace*{1\baselineskip}

\begin{figure}[H]
\centering
\caption{Função principal do código.}
\centering

\begin{lstlisting}
        program main
        call euler_crommer()
        end program main
\end{lstlisting}

\caption*{Fonte: Compilado pelo Autor.}
\label{fig:tarefa C - função principal do código}
\end{figure}

\vspace*{1\baselineskip}

\begin{figure}[H]
\centering
\caption{Implementação do método de Euler-Cromer.}
\centering

\begin{lstlisting}

        subroutine euler_crommer()
            parameter(imax=(3*1e3))
            dimension w1(0:imax), th1(0:imax)
            dimension w2(0:imax), th2(0:imax)

C       Parametros
        g = 9.81e0
        rl = 9.81e0
        rm = 1e0
        pi = acos(-1e0)
        dt = 0.04e0
        ! Parametros da força
        F0_1 = 0.5e0
        F0_2 = 0.5e0
        ome_1 = 0.75e0
        ome_2 = 0.75e0
        y_1 = 0.05e0
        y_2 = 0.05e0

C       Condições iniciais
        ! Velocidades angulares
        w1(0) = 0
        w2(0) = 0
        ! Angulo inicial
        th1(0) = 1e0
        th2(0) = 1e0 + 0.001e0

C       Cálculo
        do i = 0,imax-1
            F1 = - y_1*w1(i) + F0_1*sin(ome_1*i*dt)
            F2 = - y_2*w2(i) + F0_2*sin(ome_2*i*dt)

            w1(i+1) = w1(i) - (g/rl)*sin(th1(i))*dt + F1*dt
            w2(i+1) = w2(i) - (g/rl)*sin(th2(i))*dt + F2*dt

            th1(i+1) = th1(i) + w1(i+1)*dt
            th2(i+1) = th2(i) + w2(i+1)*dt
        end do

C       Salva os resultados
        open(unit=1,file='saida-2-12694394.txt')
        write(1,3)
        do i = 0,imax
        if (th1(i+1) .GT. 2e0*pi) then
            th1(i+1) = th1(i+1) - 2e0*pi
        else if (th1(i+1) .LT. 0e0) then
            th1(i+1) = th1(i+1) + 2e0*pi
        end if

        if (th2(i+1) .GT. 2e0*pi) then
            th2(i+1) = th2(i+1) - 2e0*pi
        else if (th2(i+1) .LT. 0e0) then
            th2(i+1) = th2(i+1) + 2e0*pi
        end if

            r1 = th1(i)
            r2 = th2(i)
            write(1,7) dt*i,w1(i),r1,w2(i),r2,r2-r1
        end do
3       format('temp,omega1,theta1,omega2,theta2,theta2-theta1')
7       format(5(F12.6,','),F12.6)
        close(1)
        end subroutine euler_crommer

\end{lstlisting}

\caption*{Fonte: Compilado pelo Autor.}
\label{fig:tarefa C - Euler Cromer código}
\end{figure}



\section*{Descrição do código}

O programa \texttt{main} tem como objetivo comparar a evolução temporal de 
dois pêndulos simples sujeitos a atrito viscoso e a uma força externa 
periódica, utilizando o método numérico de \textit{Euler-Cromer}.  
A simulação busca observar como pequenas variações nas condições iniciais 
podem gerar diferenças significativas no comportamento do sistema, 
caracterizando um regime caótico.  

\vspace*{1\baselineskip}
\noindent
O cálculo principal é realizado na subrotina \texttt{euler\_crommer}, 
chamada diretamente pelo programa principal.  

\subsection*{Parâmetros e variáveis}

Na subrotina \texttt{euler\_crommer}, é definido o número máximo de 
iterações \texttt{imax = 3×10\textsuperscript{3}} e são criados quatro 
vetores: \texttt{w1}, \texttt{th1}, \texttt{w2} e \texttt{th2}, que 
representam respectivamente as velocidades angulares ($\omega$) e ângulos 
($\theta$) dos dois pêndulos.  

As constantes físicas e os parâmetros da força externa são definidos como:

\begin{lstlisting}
g  = 9.81e0       ! Aceleração da gravidade
rl = 9.81e0       ! Comprimento do pêndulo
rm = 1e0          ! Massa
pi = acos(-1e0)   ! Valor de pi
dt = 0.04e0       ! Passo temporal

! Parâmetros da força externa
F0_1 = 0.5e0      ! Amplitude da força (pêndulo 1)
F0_2 = 0.5e0      ! Amplitude da força (pêndulo 2)
ome_1 = 0.75e0    ! Frequência da força (pêndulo 1)
ome_2 = 0.75e0    ! Frequência da força (pêndulo 2)
y_1 = 0.05e0      ! Amortecimento (pêndulo 1)
y_2 = 0.05e0      ! Amortecimento (pêndulo 2)
\end{lstlisting}

\noindent
Os dois pêndulos são idênticos em todos os parâmetros físicos, 
diferindo apenas nas condições iniciais dos ângulos.  
Ambos começam com velocidade angular nula, mas o segundo pêndulo possui um 
pequeno desvio inicial de $0{,}001$ radiano:

\begin{lstlisting}
w1(0) = 0
w2(0) = 0
th1(0) = 1.0
th2(0) = 1.0 + 0.001
\end{lstlisting}

\subsection*{Integração pelo método de Euler-Cromer}

O laço principal da simulação executa \texttt{imax} iterações, aplicando o 
método de Euler-Cromer para calcular a evolução temporal dos ângulos e 
velocidades angulares.  
A cada passo, são calculadas as forças efetivas sobre os dois pêndulos, 
compostas pelo amortecimento e pela força externa periódica:

\begin{lstlisting}
F1 = - y_1*w1(i) + F0_1*sin(ome_1*i*dt)
F2 = - y_2*w2(i) + F0_2*sin(ome_2*i*dt)
\end{lstlisting}

\noindent
Em seguida, atualizam-se as velocidades e ângulos segundo:

\begin{lstlisting}
w1(i+1) = w1(i) - (g/rl)*sin(th1(i))*dt + F1*dt
w2(i+1) = w2(i) - (g/rl)*sin(th2(i))*dt + F2*dt

th1(i+1) = th1(i) + w1(i+1)*dt
th2(i+1) = th2(i) + w2(i+1)*dt
\end{lstlisting}

\noindent
O método de Euler-Cromer utiliza a nova velocidade angular 
($\omega_{i+1}$) no cálculo da posição angular ($\theta_{i+1}$), o que 
resulta em maior estabilidade numérica para sistemas oscilatórios.  

\subsection*{Armazenamento dos resultados}

Após o cálculo, os resultados são gravados no arquivo 
\texttt{saida-2-12694394.txt}.  
Antes da gravação, é feita a correção do ângulo $\theta$ para garantir que 
permaneça dentro do intervalo $[0, 2\pi]$, evitando o acúmulo de voltas 
excedentes.  
Cada linha do arquivo contém:

\begin{itemize}
    \item o tempo ($t = i \cdot \Delta t$),
    \item a velocidade angular e o ângulo do primeiro pêndulo,
    \item a velocidade angular e o ângulo do segundo pêndulo,
    \item e a diferença angular $\Delta\theta = \theta_2 - \theta_1$.
\end{itemize}

\noindent
O formato de escrita utilizado é:

\begin{lstlisting}
7 format(5(F12.6,','),F12.6)
\end{lstlisting}

\noindent
e o arquivo é encerrado com o comando \texttt{close(1)}.  

\subsection*{Resumo}

Em síntese, o código realiza a simulação de dois pêndulos forçados e 
amortecidos com condições iniciais ligeiramente diferentes, permitindo 
analisar a divergência de trajetórias ao longo do tempo.  
Essa abordagem é típica em estudos de sistemas dinâmicos caóticos, 
onde pequenas perturbações nas condições iniciais levam a comportamentos 
macroscopicamente distintos, mesmo sob as mesmas leis de movimento.

\section*{Resultados}
Estimativa do coeficiente de Liapunov: 9.2039e-02


\begin{figure}[H]
\centering
\caption{Resultado obtidos.}
\centering
\includegraphics[width=16cm]{images/tarefa-C/saida-1-12694394_linear.png}
\caption*{Fonte: Compilado pelo Autor.}
\label{fig:tarefa C - Resultado 1}
\end{figure}

\begin{figure}[H]
\centering
\caption{Resultado obtidos.}
\centering
\includegraphics[width=16cm]{images/tarefa-C/saida-1-12694394_semilog.png}
\caption*{Fonte: Compilado pelo Autor.}
\label{fig:tarefa C - Resultado 2}
\end{figure}

Estimativa do coeficiente de Liapunov: 1.0632e-01

\begin{figure}[H]
\centering
\caption{Resultado obtidos.}
\centering
\includegraphics[width=16cm]{images/tarefa-C/saida-2-12694394_linear.png}
\caption*{Fonte: Compilado pelo Autor.}
\label{fig:tarefa C - Resultado 3}
\end{figure}

\begin{figure}[H]
\centering
\caption{Resultado obtidos.}
\centering
\includegraphics[width=16cm]{images/tarefa-C/saida-2-12694394_semilog.png}
\caption*{Fonte: Compilado pelo Autor.}
\label{fig:tarefa C - Resultado 4}
\end{figure}