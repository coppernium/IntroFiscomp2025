\chapter*{Motivação}


O estudo do movimento dos planetas é muito importante na Física, pois nos ajuda a entender como os corpos celestes interagem e se movem no espaço. A força da gravidade determina as órbitas e, ao analisar esse movimento, podemos compreender melhor o funcionamento do Sistema Solar.

Neste projeto, o objetivo é estudar tanto o caso simples de um planeta orbitando o Sol quanto situações mais complexas envolvendo vários corpos ao mesmo tempo. Usando métodos numéricos, como o método de Verlet, podemos simular diferentes condições iniciais e observar como pequenas mudanças na velocidade ou na posição modificam o formato das órbitas. Assim, é possível verificar as Leis de Kepler e entender por que algumas órbitas são circulares, outras são elípticas e algumas podem variar ao longo do tempo.

Quando mais corpos são incluídos, como a Terra e Júpiter juntos, o movimento se torna menos previsível e mais sensível a perturbações. A órbita da Terra, por exemplo, deixa de ser exatamente periódica quando a influência de Júpiter é considerada. Da mesma forma, o estudo de asteroides mostra que algumas regiões do Sistema Solar são estáveis enquanto outras apresentam lacunas causadas pelas interações gravitacionais.

Essas simulações mostram como a Física Computacional é essencial para explorar fenômenos que seriam muito difíceis de resolver apenas com contas analíticas. Mesmo sistemas que seguem leis simples podem apresentar comportamentos complexos. Por isso, a computação é uma ferramenta fundamental para investigar e compreender melhor esse tipo de problema.